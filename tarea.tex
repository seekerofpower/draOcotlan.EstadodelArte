\documentclass[10pt,letterpaper]{article}
\usepackage[utf8]{inputenc}
\usepackage{amsmath}
\usepackage{amsfonts}
\usepackage{vmargin}
\usepackage{amssymb}
\usepackage{graphicx}
\usepackage[spanish]{babel}
\bibliographystyle{apalike} 
\usepackage[numbered,framed]{matlab-prettifier}
\usepackage{filecontents}

\spanishdecimal{.}
\begin{document}
\begin{titlepage}




\begin{center}
\includegraphics[scale=.3]{escudo.png}}  \\[1 cm]
\textbf{\LARGE UNIVERSIDAD POLITÉCNICA DE PACHUCA}\\[0.2cm]
\vspace{30pt}
\textbf{\large Maestría en Tecnologías de la Información y Comunicaciones}\\[1 cm]
\end{center}
\par
\vspace{30pt}


\begin{center}
\text{\large Materia:   Proyecto de Tesis}\\
\text{\large Actividad:   Citas}
\end{center}


\begin{center}
\vspace{30pt}
\textbf{\large Nombre del Alumno:}\\[0.1cm]
\Large {Benjamin Oribe Mendieta} \\[0.1cm] 

\textbf{\large Nombre del Profesor:}\\[0.1cm]
\Large {Dra.Ocotlán Díaz Parra} \\[0.1cm] 

\end{center}
\end{titlepage}
\section{Blockchain}

Normalmente se considera a Blockchain como una sola tecnología, pero en \cite{halaburda2018blockchain} se advierte como características propias de la tecnología como la encriptación, el manejo de transacciones distribuido y los contratos inteligentes son tecnologías importantes por si mismas que pueden cambiar varias industrias.\\


Actualmente se desarrollan varias implementaciones alternativas al método utilizado por Bitcoin para el manejo de inconsistencia por tolerancia Bizantina  como en \cite{10.1145/3152824.3152830} donde crearon un servicio con el sistema Hyperledger Fabric\cite{bessani2014state}.\\

En \cite{puthal2018blockchain} presentan un análisis de las posibilidades de blockchain para solventar problemas de seguridad y validación de transacciones.\\

\subsection{Seguridad}

Blockchain no es inmune a ataques, en \cite{hasanova2019survey} y \cite{apostolaki2017hijacking}se muestran las vulnerabilidades de esta tecnología como el ataque de DDOS a Ethereum en 2016\cite{atzei2016survey}. En el trabajo de \cite{destefanis2018smart} se describe como el programar de manera insegura los \textit{smart contracts} de Etherewum(\cite{buterin2016ethereum}) llevo a un congelamiento de 500 Ethers por medio de la wallet \textit{Parity}\cite{Parity} en 2017.\\

Los sistemas criptográficos de firma digital se consideran vulnerables ante ataques por métodos cuánticos, en \cite{li2018new} se presentan una opción por medio del algoritmo de Arboles Bonsai\cite{cash2010bonsai}  y el algoritmo de \textit{RandBasis} para\cite{sciabarra2013ayn} crear wallets no deterministas para aplicaciones de blockchain. \cite{xu2016blockchains} hace un listado de al menos 4 tipos de amenazas a los sistema blockchain.\\

En \cite{gervais2016security} desarrollan un framework capaz de analizar la vulnerabilidad de varias plataformas blockchain a diferentes grupos de ataques, con interés en el ataque de Eclipse(\cite{wust2016ethereum}). \\

En \cite{wust2016ethereum} se hace un análisis a los ataques por medio de Eclipse a la plataforma Ethereum y en \cite{nayak2016stubborn} se maneja el crecimiento de estos ataques en general y sobre el crecimiento de la minería no egoísta de cripto monedas.\\

Las cripto monedas es uno de los mas populares sectores de implementación de blockchain, muchas nuevas se unen a bitcoin\cite{nakamoto2008bitcoin}, en \cite{chan2017statistical}, se hace un análisis las estadísticas principales de las principales monedas en el mercado. En \cite{kyriazis2019survey} se busca ver si el precio de las cripto monedas principales así como esta se relaciona con la eficiencia de cada una. Por su parte \cite{ivanov2018technical} genera una plataforma de comparación para analizar los aspectos técnicos de las plataformas principales de cripto monedas para determinar la mejor para resolver cada problema específico.\\



\subsection{Mejoras}

Existen muchas cripto monedas en el mercado y atrae a muchas organizaciones, en \cite{libra2019} Facebook presenta Libra, una cripto moneda para el pago de servicios que busca ser soportada por organizaciones gubernamentales y bancarias, en \cite{taskinsoy2019facebook} se intenta advertir si Libra puede cumplir la idea de una cripto moneda para las transacciones de todos los días.\\ 

Aunque uno de los pilares de las plataformas Blockchain es la transparencia de las operaciones, también existen investigaciones como la de \cite{feng2019survey} donde analiza protección de la privacidad en sistemas Blockchain. En \cite{esposito2018blockchain} se analiza si Blockchain puede ser una opción para la privacidad en registros médicos compartidos. \cite{jiang2019ptas} analizan la privacidad de un sistema de llave publica para clientes livianos. En \cite{lemieux2016trusting}se advierte si la tecnología blockchain actual es capas de mantener un sistema de registro conmfiable usando la plataforma como ejemplo Bitcoin y Factom(\cite{snow2015factom}).\\ 

La obtención del proof-of-work es uno de los pasos de Blockchain donde existe espacio para mejorar en  \cite{hazari2019parallel} presentan un método para hacer el proceso de minado de manera paralela. En \cite{kang2018incentivizing} se genera un sistema para la rápida propagación del consenso de nodos en sistemas basados en Proof-Stake(\cite{kiayias2017ouroboros}) utilizando un juego de Stackelberg(\cite{zhang2009stackelberg}) para maximizar la ganancia de los mineros. \\

En \cite{saleh2020blockchain} se presenta un modelo económico basado en proof-of-stake donde se puede obtener el consenso de manera general al evitar que los "accionistas" retrasen el consenso. En \cite{liu2019fork} se busca crear un modelo consenso hibrido basado en el  Proof-of-activity  de \cite{bentov2014proof} y la implementación de \cite{pass2017hybrid}  que permite un modelo libre de \textit{forks}.\\

También \cite{yoo2018promoting} advierten que la estrategia no cooperativa de Ethereum ayuda a que la generación de bloques vacíos (sin transacciones) sean mas comunes, lo cual reduce la velocidad de proceso de transacciones y proponen un sistema con una estrategia cooperativa para solucionar este problema.\\

Existe un lado ecológico que se debe tomar en cuenta para los sistemas Blockchain,\cite{li2019energy}  hace el análisis de la energía requerida en un año para minar la cripto moneda \textit{Monero} \cite{Monero}. Por su parte \cite{krause2018quantification} analiza los factores que influyen en marca de carbono de la minería de cripto monedas.\\


\subsection{Aplicaciones}


Blockchain ha atraído a varios mercados, de esta manera, \cite{cai2018decentralized} han hecho un survey de las posibles aplicaciones que se verían beneficiadas del uso de esta arquitectura/plataforma. Así mismo \cite{singh2020blockchain} analizan las posibilidades y retos inherentes en formalizar los contratos inteligentes por medio de blockchain.\\
 
Blockchain y su implementación es de mucha importancia para las aplicaciones de la industria, en \cite{mohamed2019applying} se muestra un listado de los puntos en los cuales la tecnología blockchain tendría un impacto para la implementación de la industria 4.0 (\cite{lasi2014industry}).\\

En \cite{yi2019securing} se busca implementar un sistema seguro para sistemas de voto electrónico.\\

La implementación de Internet of Things requiere de unión de varias tecnologías para poder implementarse cabalmente, en \cite{dai2019blockchain} y \cite{wang2019survey} se hace un listado de como blockchain puede ayudar en este proceso. En el proyecto de \cite{alphand2018iotchain} se propone un concepto IoTChain, una combinación de la arquitectura OSCAR(\cite{vuvcinic2015oscar}) y el framework de autorización ACE (\cite{seitz2017authentication} para manejo de clientes de IOT entre nodos finales.\\

Para reducir la letencia ente dispositivos parta IOT en \cite{sharma2018blockchain} proponen una arquitectura híbrida por software que maneje secciones centralizadas y descentralizadas para su implementación sen ciudades inteligentes.\\

En el trabajo de \cite{khan2018iot} se analizan los aspectos a tomar en cuenta a nivel de seguridad de implementar sistemas basados en Blockchain para IOT.


En \cite{motohashi2019secure} analiza y diseña un sistema para manejar sistema de Salud Móvil (mHealth) basado en blockchain. En el mismo ámbito \cite{cao2019cloud} diseñan un sistema basado en la nube para manejo electrónico de registros médicos electrónicos.\\

Dentro del survey de \cite{rabah2018convergence} se analiza los puntos en que tecnologías como blockchain convergen cono IoT, el análisis de Big Data y la Inteligencia Artificial.

En campo de la robótica \cite{ferrer2018blockchain} presenta como blockchain puede solventar problemas en los campos de seguridad, toma de desiciones, diferenciación de comportamientos y modelos de negocios para el campo de la robótica de enjambres.\\

\section{Hashchain}

Las funciones hash para el manejo de firmas digitales, y es necesario tener implementaciones de las mas conocidas para varios dispositivos. \cite{osvik2012fast} presentan un listado de eficiencia de las implementaciones de funciones hash en para sistemas embebidos, mientras \cite{kyu2002efficient} desarrollan una función hash implementación para procesadores IPSEC, \cite{cheng2018efficient} desarrollan una implementación de SHA-512 para procesadores ARV, comunes en las tarjetas arduino. En \cite{kim2018optimized} se desarrolla una implementación de SHA-3 para sistemas en un chip (SOC) para procesadores ARM.\\



En \cite{wang2019message} se desarrolla un protocolo de identificación de mensajes por medio de una función hash cuántica, basada en quantum walks (\cite{li2013discrete}).\\

El IOT también busca en la firma por medio de hash chains(\cite{lamport1981password} una manera segura de firmar información, en \cite{alshahrani2019secure} crean un a identidad temporal basado en hash chain para identificación mutua y control automatizado para hogares inteligentes.\\

En \cite{jeong2020multi} se desarrolla un sistema para recuperación de información multimedia para sistema distribuidos para IoT basado en hash chains.\\

En \cite{tariq2018detection} se utilizan hash chains para analizar la entrada de datos falsos en redes de sensores inalámbricos tomando en cuanta sus limitadas capacidades de procesamiento. A si mismo |cite{lu2018research} utiliza hash chains para resolver el problema de falta de autentificación en el modo de trasnmisión de las redes DNP3(\cite{clarke2004practical}). \\

Una de las aplicaciones mas comunes de hash chains es la generación de passwords temporales como se muestras en la implementación de \cite{park2018one}.\\



\bibliography{referencias}
\end{document}
